\begin{abstract}
Følgende opgave undersøger sikkerheden ved elliptisk kurvekryptografi. 
Der redegøres for elliptiske kurver hvorefter gruppestrukturen på dem bliver præsenteret. Herefter gives der en udledning for additionen af punkter i gruppen hvorefter begrebet skalarmultiplikation introduceres. Her bliver tiden for at lave $n$ additioner reduceret til $O(log_{2}n)$ i stedet for $O(n)$. Dernæst bliver elliptiske kurver på endelige legemer defineret, så en computer kan regne på dem. Herefter kigges der på hvordan elliptiske kurver kan anvendes indenfor kryptografi, hvor begrebet \textit{assymetrisk kryptering} introduceres. Algoritmen for et \code{Elliptic Curve Diffie Hellmann key exchange} gennemgås hvor der redegøres for det diskrete logaritmeproblem. Egen data fra en simulation af offentlig nøgle-generering sammenlignes med privat-nøgle brydning. Det ses at krypteringstiden stiger lineært, mens tiden for at bryde krypteringen stiger eksponentielt for bitlængden af privatnøglen. Der designes et system til elliptisk kurvekryptografi, hvor der med pseudokode redegøres for addition på en elliptisk kurve samt skalarmultiplikation. Sikkerheden i det implementerede system bliver vurderet som sikkert så langt tid, at man kan stole på den anden part. Til sidst sammenlignes elliptisk kurvekryptosystemer med RSA, hvor det konkluderes at en meget kortere key-længde for Elliptisk kurve kryptografi er nødvendig for samme "sikkerhed" som i et RSA-kryptosystem.
\end{abstract}