\section{Bilag}

\subsection{Elliptisk Kurvekryptografi kode}
\label{sec:code}
\begin{python}
INF_POINT = None 

class EllipticCurve:
	def __init__(self, p, A, B, G, n):
		self.p = p # prime number
		self.A = A # Coefficient
		self.B = B # Coefficient
		self.G = G # Generator point
		self.n = n # order of subgroup

	def is_on_curve(self, P):
		"""" Check whether point is located on curve """
		if P is INF_POINT:
			return True

		x, y = P
		
		return (y * y - x * x * x - self.A * x - self.B) % self.p == 0	

	def add(self, P, Q):
		""" Returns the new point of P+Q according to the group law defined in pseudocode"""
		if P == INF_POINT:
			return Q
		if Q == INF_POINT:
			return P
		
		x1, y1 = P
		x2, y2 = Q

		if x1 == x2 and y1 != y2:
			return INF_POINT

		if x1 == x2:
			slope = (3 * x1 * x1 + self.A) * self.inverse_modp(2 * y1)			
		else:
			slope = (y1 - y2) * self.inverse_modp(x1 - x2)

		x3 = slope * slope - x1 - x2
		y3 = y1 + slope * (x3 - x1)
		new_point = (x3 % self.p, -y3 % self.p)

		return new_point

	def double(self, P):
		"""" Get double of point. Do the addition P+P"""
		return self.add(P, P)

	def negative(self, P):
		if P is INF_POINT:
			return INF_POINT
		
		x, y = P
		neg_point = (x, -y % self.p)

		return neg_point

	def mult(self, n, P):
		""" add P to itself n times"""	
		if n % self.n == 0 or P is INF_POINT:
			return INF_POINT
		if n < 0:
			return self.negative(self.mult(-n, P))
		new_point = INF_POINT
		temp_point = P
		while n:
			if n & 1: #bitwise returns 1 if the last bit is 1.
				new_point = self.add(new_point, temp_point)
			temp_point = self.double(temp_point)
			n >>= 1 # Shift binary number 1 bit
		return new_point

	def brute_force(self, public_key):
		temp_point = self.G
		for i in range(1, self.p):
			if temp_point == public_key:
				return(i)
			temp_point = self.mult(i+1, self.G)

	def discriminant(self):
		"""Returns discriminant of curve 
		if D != 0 then elliptic curve is valid"""
		return ((4*self.A**3) + (27*self.B**2)) % self.p != 0

	def inverse_modp(self, n):
		"""Returns the inverse of n modulo p.
			This function returns the only integer x such that (x * n) % p == 1.
			n must be non-zero and p must be a prime.
			source: http://anh.cs.luc.edu/331/notes/xgcd.pdf
		"""
		if n == 0:
			raise ZeroDivisionError('division by zero')
		if n < 0:
			return self.p - self.inverse_modp(-n)

		s, old_s = 0, 1
		t, old_t = 1, 0
		r, old_r = self.p, n

		while r != 0:
			quotient = old_r // r
			old_r, r = r, old_r - quotient * r
			old_s, s = s, old_s - quotient * s
			old_t, t = t, old_s - quotient * t

		_, x, _ = old_r, old_s, old_t

		return x % self.p
\end{python}

\subsection{Omregning fra sekunder til år}
\label{section:omregning}
En computer laver $2^{30}$ operationer pr. sek. Vi løser ligningen hvor $x$ er antal sekunder.
$$2^{30} \cdot x = 2^{100}$$
$$x=\frac{2^{100}}{2^{30}}=2^{100}-2^{30}=2^{70}$$
Vi ved at det tager $2^{70}$ sekunder. Der er $3,154\cdot 10^7$ sekunder på et år. Hermed kan vi opskrive følgende.
$$\frac{2^{70}\cdot s}{3,154 \cdot 10^7 \cdot \frac{s}{aar}} = 3,74\cdot 10^{13}\cdot aar$$

\subsection{Grafer}
Alle grafer er fremstillet med python biblioteket matplotlib. Nedenunder kan koden for fremstillingen ses.

\subsubsection{Figur 1.}
\label{section:figur_1}
\begin{python}
def ecc_r(x):
    return x**3 - 2*x + 8

fig, ax = plt.subplots(dpi=100)
y, x = np.ogrid[-5:5:100j, -5:5:100j]
plt.contour(x.ravel(), y.ravel(), y**2 - ecc_r(x), [0])
plt.grid()
plt.axhline(y=0, color='k')
\end{python}

\subsubsection{Figur 2.}
\label{section:figur_2}
\begin{python}
# Points P and Q on E
P = np.array((-2, np.sqrt(ecc_r(-2))))
Q = np.array((0, np.sqrt(ecc_r(0))))

ax.scatter(*P)
ax.annotate('P', (P[0], P[1]+0.25), c='b')

ax.scatter(*Q)
ax.annotate('Q', (Q[0], Q[1]+0.25), c='b')

# Intersecting line
t = np.arange(-1, 3, 0.001)
line = np.array([(Q-P) * t_i + P for t_i in t])
ax.plot(line[:,0], line[:,1], linestyle='--', c='black', linewidth=1)

# Find points where the line intersects E
for l in line:
    if ecc_r(l[0]) > 0:
        if np.isclose(l[1], np.sqrt(ecc_r(l[0])), rtol=0.0001):
            R = l

# Plot the last point that intersected
ax.scatter(*R)
ax.annotate('R', (R[0]+0.1, R[1]-0.45), c='b')

# Draw vertical line, and the new point which is our sum P+Q
ax.axvline(R[0], linestyle='--', c='black', linewidth=1)
ax.scatter(R[0], -R[1])
ax.annotate('(-R) P+Q', (R[0]+0.1, -R[1]+0.25), c='b')

# Draw the point at infinity to have 3 points of intersection for vertical line
ax.scatter(R[0], ax.get_ylim()[1])
ax.annotate(r'$\mathcal{O}$', (R[0]+0.1, ax.get_ylim()[1] - 0.35), c='b')
fig
\end{python}

\subsubsection{Figur 3.}
\label{section:figur_3}
\begin{python}
fig = plt.figure(dpi=100)

# The point P
P = np.array((-1, np.sqrt(ecc_r(-1))))

plt.scatter(*P)
plt.annotate('P', (P[0]-0.1, P[1]+0.25), c='b')
plt.scatter(P[0]+0.1, P[1])
plt.annotate('Q', (P[0]+0.1, P[1]-0.5), c='b')


# Plot ECC
y, x = np.ogrid[-5:5:1000j, -5:5:1000j]
plt.contour(x.ravel(), y.ravel(), y**2 - ecc_r(x), [0])

# Dervative to find tangent point
def d_dx_e(x):
    n = 3 * x ** 2 - 5
    d = 2 * np.sqrt(x**3 - 5 * x + 8)
    return n / d

# Plot the tangent to P
t = np.arange(-4, 4, 0.01)
def y_intercept(x_i, m):
    return m * (x_i - P[0]) + P[1]

line = y_intercept(t, d_dx_e(P[0]))
plt.plot(t, line, linestyle='--', c='black', linewidth=1)

# Get the intersection points of E with tangent line
temp = [ecc_r(t_i) if ecc_r(t_i) > 0 else 0 for t_i in t]
idx = np.argwhere(np.diff(np.sign(np.sqrt(temp) - line))).flatten()[-1:][0]

R = t[idx], line[idx]
plt.scatter(*R)
plt.annotate('R', (R[0]+0.1, R[1]-0.45), c='b')

# Draw vertical line
plt.axvline(t[idx], linestyle='--', c='black', linewidth=1)

plt.scatter(R[0], -R[1])
plt.annotate('P+P', (R[0]+0.1, -R[1] + 0.25), c='b')


# Draw the point at infinity to have 3 points of intersection for vertical line
plt.scatter(R[0], ax.get_ylim()[1])
plt.annotate(r'$\mathcal{O}$', (R[0]+0.1, ax.get_ylim()[1] - 0.35), c='b')
\end{python}

\subsubsection{Figur 4.}
\label{section:figur_4}
\begin{python}
# Order
p = 487
x = np.array(range(0, p))

A = -5
B = 8

x2 = x**2 % p
y2 = ecc(x, p, A, B)

# Compute the sqrt of the elliptic curve for plotting
y = sqrt_f(x, y2, p)

fig = plt.figure(dpi=100)
for y_p in y:
    [plt.scatter(y_p[0], i, c='b') for i in y_p[1:]]
#plt.title('p=487   y^2=x^3-5x+8 (mod p)')
\end{python}

\subsubsection{Figur 6.}
\label{section:figur_6}
\begin{python}
# M-221 https://eprint.iacr.org/2013/647.pdf
p = 0x1FFFFFFFFFFFFFFFFFFFFFFFFFFFFFFFFFFFFFFFFFFFFFFFFFFFFFFD
A = 0x01c93a
B = 0x01
G = (0x04, 0x0f7acdd2a4939571d1cef14eca37c228e61dbff10707dc6c08c5056d)
n = 0x040000000000000000000000000015A08ED730E8A2F77F005042605B

e = EllipticCurve(p, A, B, G, n)

gen_times = []
break_times = []

iterations = 15
for i in range(1, iterations):
    private_key = 2**i
    startgen = timer()
    public_key = e.mult(private_key, G)
    endgen = timer()
    gen_times.append(endgen - startgen)

    startbreak = timer()
    breaked_public_key = e.brute_force(public_key)
    endbreak = timer()
    break_times.append(endbreak - startbreak)
    print(private_key)

print("gen: ", gen_times)
print("")
print("break: ", break_times)

x = range(1,iterations)
slope, intercept, r_value, p_value, std_err = stats.linregress(x, gen_times)

coef = np.polyfit(x,gen_times,1)
poly1d_fn = np.poly1d(coef) 
fig, ax1 = plt.subplots()


ax1.set_xlabel('Binaert tal laengde')
ax1.set_ylabel('tid (s)')
ax1.plot(x, gen_times, 'yo', x, poly1d_fn(x), '--k')
plt.title('Bits i private_key vs tid for public-key generering ' + f'{poly1d_fn}' + f'   R^2={"{:.2f}".format(r_value)}')
fig = plt.figure(dpi=300)
plt.show()

slope, intercept, r_value, p_value, std_err = stats.linregress(x, np.log(break_times))

coef = np.polyfit(x, np.log(break_times),1)
poly1d_fn = np.poly1d(coef) 

fig, ax1 = plt.subplots()
ax1.set_xlabel('Binaert tal laengde')
ax1.set_ylabel('tid (s) paa logaritmisk skala')
color = 'tab:red'
ax1.tick_params(axis='y', labelcolor=color)

ax1.plot(x, np.log(break_times), 'ro', x, poly1d_fn(x), '--k')

ax2 = ax1.twinx()
color = 'tab:blue'
ax2.set_ylabel('tid (s)')
ax2.plot(x, break_times, 'bo')
ax2.tick_params(axis='y', labelcolor=color)

plt.title('Bits i private_key vs tid for private-key breaking' + f'{poly1d_fn}' + f'   R^2={"{:.2f}".format(r_value)}')
fig = plt.figure(dpi=300)

plt.savefig('break.png')
plt.show()
\end{python}


