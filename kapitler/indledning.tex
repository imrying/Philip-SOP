\section{Indledning}

Op indtil 1970'erne var alt kryptering baseret på \textit{symmetriske-funktioner}. Dette betød, at hvis to parter skulle kunne kommunikere sikkert, var de nødt til at møde fysisk først og blive enige om en \textit{fælles hemmelighed}. Op indtil det 20. århundrede var symmetrisk kryptering tilfredsstillende, men på baggrund af anden verdenskrig og internettes fødsel steg motivationen for at lave et system, hvor man kunne kommunikere sikkert uden at kende hinanden på forhånd.
Denne type kryptografi er kendt som \textit{assymetrisk-} eller \textit{public-key} kryptering og går ud på, at man har en offentlig- og privat nøgle. Den grundlæggende idé er, at man bruger den offentlige nøgle til at \textit{kryptere} beskeder, mens den private nøgle bruges til at \textit{dekryptere} beskeden og holdes altså privat (\cite{seanriley2017}). For at lave et sådan system har man brug for et sæt af algoritmer, hvor det er nemt at gå den ene vej, men meget svært at gå den anden. Den første, og til dags dato, mest anvendte er RSA-kryptering. Dens sikkerhed bygger på, at det er nemt at gange to store primtal sammen, men at faktorisere produktet til dets to primtal er svært \textit{(Diskrete logaritmeproblem)}. Problemet ved RSA-kryptering er dog, at der i dag er fundet algoritmer, som kan løse det diskrete logaritmeproblem på smartere måder en rent gætværk.  Dette har gjort, at RSA-nøglers længde har været nødt til at blive længere og længere efterhånden som computere er blevet hurtigere. Et nyt system som bygger på elliptiske kurver, har dog potentiale til at overtage for RSA-kryptering, da det diskrete logaritmeproblem for elliptisk kurvekryptografi, viser sig sværere at løse. 
I følgende opgave undersøges elliptisk kurvekryptografi fra definitionen af en elliptisk kurve, helt til implementering af et elliptisk kurvekryptosystem. Til sidst vurderes sikkerheden af det implementerede system, hvorefter det sammenlignes med RSA. 